\documentclass{article}
\usepackage[margin=1.0in]{geometry}
\usepackage{amssymb, amsmath, amsfonts, graphicx}
\graphicspath{ {/} }

\DeclareMathOperator{\for}{for}
\DeclareMathOperator{\lcm}{lcm}

\begin{document}
\noindent Davis Lister and Jacob Fisher
\newline May 16, 2018

\begin{abstract}
We will explore the properties of elliptic curves as an abelian group, as well as investigating some applications of the group to integer factorization problems and public-key cryptography.
\end{abstract}

We will define an elliptic curve over a field $K$ as an equation of the form 
$$y^2=x^3+ax+b,$$ 
where $a,b \in K$ and the discriminant $-16(4a^3+27b^2) \neq 0$. We may impose an abelian group structure on the set
$$E(K)=\lbrace(x,y) \in K \times K: y^2=x^3+ax+b\rbrace \cup \lbrace \mathcal{O} \rbrace$$
of $K$-rational points on an elliptic curve $E$ over the field $K$. Often, we will implement the group structure on elliptic curves over fields of the form $\mathbb{Z}/p\mathbb{Z}$, though they are not limited to such fields. The group structure may also be imposed over $\mathbb{R}$, for instance.

\indent Let $E$ be an elliptic curve over a field $K$, with the equation $y^2=x^3+ax+b$. We begin by defining the binary operation $+$ on $E(K)$ such that, for $P_1,P_2 \in E(K)$, $P_1 + P_2=R \in E(K)$.  We will define the $+$ operation as follows:
\begin{enumerate}
\item If $P_1=\mathcal{O}$ set $R=P_2$ or if $P_2=\mathcal{O}$ set $R=P_1$, terminate and return $R$. Otherwise write $P_1=(x_1,y_1)$ and $P_2=(x_2,y_2)$.
\item If $x_1=x_2$ and $y_1=-y_2$, set $R=\mathcal{O}$, terminate and return $R$.
\item If $P_1=P_2$, set $\lambda=\frac{3x_1^2+a}{2y_1}$. Otherwise, set $\lambda=\frac{y_1-y_2}{x_1-x_2}$.
\item Let $x_3=\lambda^2-x_1-x_2$, $\nu=y_1-\lambda x_1$. Then, $R=(x_3,-\lambda x_3-\nu)$. Terminate and return $R$.
\end{enumerate} 

\indent It is clear that the identity element in the proposed group $(E(K),+)$ is $\mathcal{O}$, by item one of the above definition. Additionally, item two implies that for some $P \in E(K)$, where $P=(x,y)$, $P^{-1}=(x,-y)$, since $P+P^{-1}=\mathcal{O}$. Furthermore, the definition implies that the group is abelian since at each step one may substitute $P_1$ for $P_1$ while leaving $P_1+P_2=R$ unchanged. In order to prove that the group $(E(K),+)$ is closed, we will interpret the $+$ operation geometrically.

\indent The group operation $+$ can be interpreted as intersecting the secant line $L$ drawn between two points $P_1=(x_1,y_1)$ and $P_2=(x_2,y_2)$ where $P_1,P_2 \in E(K)$ and $x_1 \neq x_2$ with the curve $S$, which is given by the equation $y^2=x^3+ax=b$. After finding $P_3=(x_3,y_3) \in L \cap S$, we reflect $P_3$ over the x-axis to arrive at $P_1+P_2$. In the case that $x_1=x_2$, the resulting line will either be the line tangent to $S$ at $x_1$, or it will be the vertical secant line connecting $(x_1,y_1)$ and $(x_2,y_2)$. We will address the first case later on. In the second case, the secant line will intersect $S$ at $\mathcal{O} \in E(K)$, which when reflected over the x-axis remains $\mathcal{O}$.

\indent Let $L$ be the line drawn between $P_1,P_2$ where $P_1,P_2$ are on $S$. $L$ is given by the equation
$$y=y_1+(x-x_1)\lambda.$$
Substituting the equation for $L$ into the equation for $S$ we obtain
$$(y_1+(x-x_1)\lambda)^2=x^3+ax+b.$$
By simplifying we arrive at
$$x^3-\lambda^2 x^2+2\lambda x_1x-2y_1x+ax-y_1^2+2y_1x_1+b=0.$$
For $A,B\in\mathbb{R}$, the above equation can be written as
$$x^3-\lambda^2 x^2+Ax+B=0.$$
Since $P_1,P_2 \in L \cap S$, the polynomial above will have $x_1$ and $x_2$ as solutions. By the fundamental theorem of algebra
$$0=x^3-\lambda^2 x^2+Ax+B=(x-x_1)(x-x_2)(x-x_3).$$
By expanding the factored form of the polynomial we obtain
$$0=(x-x_1)(x-x_2)(x-x_3)=x^3-(x_1+x_2+x_3)x^2+(x_1x_2-x_1x_3-x_2x_3)x+x_1x_2x_3.$$
Therefore, we can deduce that
$$x_3=\lambda^2-x_1-x_2.$$
From the equation for L, we obtain that
$$y_3=y_1+(x_3-x_1)\lambda=\lambda x_3+\nu$$
where $\nu=y_1-\lambda x_1$. In the case that $P_1=P_2$, $\lambda=\frac{3x_1^2+a}{2y_1}$. As such, $L$ will intersect $S$ at two points and the polynomial $(x-x_1)(x-x_2)(x-x_3)=0$ will have a double root $x_1=x_2$. However, the group structure will still be maintained since there will still be a unique solution $x_3$, and the above proof does not rely on the precise value of  $\lambda$, only the structure of the equation for $S$. Thus, $(E(K),+)$ is closed.

\indent In some cases, the group represented by an elliptic curve can have a cyclic structure. In the case of the curve $S$ represented by $y^2=x^3+x+1$ over $\mathbb{Z}/5\mathbb{Z}$, the group has order $9$ and the generator element $G \in S(\mathbb{Z}/5\mathbb{Z})$ is the point $(0,1)$.

\indent We return to address the requirement that the discriminant, $\Delta=-16(4a^3+27b^2) \neq 0$. We we wish to show that $\Delta=0$ if and only if an elliptic curve $S$ is singular. A singular curve poses problems for a group structure because at the singular point the derivative is not well defined, which would cause the addition the singular point to itself to not be well defined. For any polynomial $f(x_1,x_2,\dotsc)$ at the singular point $P_0$
$$\frac{\partial f}{\partial x_1}(P_0)=\frac{\partial f}{\partial x_2}(P_0)=\cdots=0.$$

\indent First, we wish to show that if $S$ is singular, then $\Delta=0$.
Let $f(x,y)=y^2-(x^3+ax+b)$. Suppose $S$ has a singular point at $P_0=(x_0, y_0)$. Therefore,
$$\frac{\partial f}{\partial x} y_0^2-(x_0^3+ax_0+b)=-3x_0^2-a=0 \Longrightarrow a=-3x_0^2,$$
$$\frac{\partial f}{\partial y} y_0^2-(x_0^3+ax_0+b)=-2y_0=0 \Longrightarrow y_0=0.$$
Since $y_0=0$, all singular points will be roots of $y^2=x^3+ax+b$. Observe,
$$0=x_0^3-3x_0^3+b \Longrightarrow b=2x_0^3.$$
Thus,
$$\Delta=-16(4(-3x_0^2)^3+27(2x_0^3)^2)=0.$$

\indent We wish to show that if $\Delta=0$, then $S$ is singular. First we will prove a lemma: if $\Delta=0$, $y=x^3+ax+b$ has a double root $x_0$. Note that
$$-16(4a^3+27b^2)=0 \Longrightarrow b = \sqrt{\frac{-4a^3}{27}}.$$
Furthermore, observe that given the above our equation becomes
$$y=x^3+ax+\sqrt{\frac{-4a^3}{27}}.$$
The roots of the above equation are
$$x_1=\frac{a}{\sqrt{3}\sqrt[6]{-a^3}}-\frac{\sqrt[6]{-a^3}}{\sqrt{3}}, x_2=\frac{i\sqrt{3}\sqrt[3]{-a^3}+\sqrt[3]{-a^3}+i\sqrt{3}a-a}{2\sqrt{3}\sqrt[6]{-a^3}}, x_3=\frac{-i\sqrt{3}\sqrt[3]{-a^3}+\sqrt[3]{-a^3}-i\sqrt{3}a-a}{2\sqrt{3}\sqrt[6]{-a^3}}.$$
Note that
$$b=\sqrt{\frac{-4a^3}{27}} \Longrightarrow a<0 \for b \in \mathbb{R}.$$
Therefore, our second two solutions become identical
$$x_0=\frac{\sqrt[3]{-a^3}-a}{2\sqrt{3}\sqrt[6]{-a^3}}.$$
Thus $y=x^3+ax+b$ has a double root $x_0$.
By the lemma, we can deduce that one of the roots of $y=x^3+ax+b$ is a root of its derivative, $y'=3x^2+a$. Recall that $f(x,y)=y^2-(x^3+ax+b)$. Therefore,
$$f(x_0,0)=0^2-(x_0^3+ax_0+b)=0,$$
$$\frac{\partial f}{\partial x} (x_0,0)=-3x_0^2-a=0,$$
$$\frac{\partial f}{\partial y} (x_0,0)=2(0)=0.$$
Thus S is singular. Furthermore, we can conclude that $S$ is singular if and only if $\Delta=0$. 

\indent An important application of the group $(E(K),+)$ is integer factorization. We will begin by discussing Pollard's $p-1$ algorithm for integer factorization as a preface to our discussion of Lenstra elliptic curve factorization.

\indent We will define the concept of being power smooth. Let $B$ be a positive integer. The prime factorization of an integer $n=\prod p_{i}^{e_{i}}$. If $\forall i$ $p_{i}^{e_{i}} \leq B$, $n$ is $B$-power smooth. Now we will attempt to find a nontrivial factor of a large positive integer $N$ using the Pollard p-1 method. Let us choose a positive integer $B$. Suppose that there is a prime factor $p$ of $N$ such that $p-1$ is $B$-power smooth. Let us choose $a > 1$ such that p does not divide $a$. Often we will choose $a=2$ for convenience. By Fermat's Little Theorem
$$a^{p-1} \equiv 1 \mod p.$$
Let $m = \lcm (1,2,3,\dotso,B)$. Since $p-1$ is $B$-power smooth, 
$$p-1 \mid m \Longrightarrow p \mid \gcd (a^m-1,N) > 1.$$

\indent We will now explore an example of integer factorization with Pollard's $p-1$ algorithm. Let $N=5917$ and let $B=5$. Therefore, 
$$m=\lcm (1,2,3,4,5)=60.$$
Note that $2^{60}-1=3416 \mod 5917$. Thus,
$$\gcd (2^{60}-1, 5917)=\gcd (3416, 5917)=61.$$
Therefore, 61 is a factor of 5917.

\indent We will now define Lenstra's algorithm. Given a positive integer $N$ and a bound $B$, this algorithm attempts to find a nontrivial factor $p$ of $N$. We will define the algorithm with the following steps:
\begin{enumerate}
\item Compute $m=\lcm(1,2,\dotso,B)$. 
\item Choose a random $a \in \mathbb{Z}/N\mathbb{Z}$ such that $4a^3+27 \in \mathbb{Z}/N\mathbb{Z}^*$. Thus, $P=(0,1)$ is a point on the elliptic curve $y^2=x^3+ax+14$ over $\mathbb{Z}/N\mathbb{Z}$.
\item Attempt to compute $mP$ using the $+$ operation for the group $(E(K),+)$. If at some point we cannot compute a sum of points because $\gcd(x_1-x_2,N) \neq 1$ (where $x_1-x_2$ is the denominator of the slope expression we compute in order to execute the $+$ operation), compute and return $\gcd(x_1-x_2,N)$ if $\gcd(x_1-x_2,N) \neq N$. If some point $kP=\mathcal{O} \for k \leq m$, terminate and output, "Fail." Additionally, if $mP$ can be computed using the $+$ operation, output, "Fail."
\end{enumerate}

\indent The advantage of the Lenstra method is that if the algorithm fails, we may choose a different elliptic curve and repeat the algorithm. In Pollard's $p-1$ we always work with the group $\mathbb{Z}/N\mathbb{Z}^*$, which has order p-1. However, in the Lenstra method we work with many different groups $E(\mathbb{Z}/N\mathbb{Z})$, which will have order $p+1\pm s$, for some nonnegative integer where $s<2\sqrt{p}$ by Hasse's theorem. Therefore, we will have more flexibility since the order of our group is not always fixed to $p-1$.

\indent We will now explore an example of integer factorization with Lenstra's algorithm. Let $N=5959$ and let $B=8$. Therefore, $m=\lcm(1,2,\dotso,8)=840$. We will randomly choose $a=6$, which produces the curve $S$ defined by the equation $y^2 \equiv x^3+6x+1 \mod 5959$, containing the point $P=(0,1)$. We now attempt to compute $840P$. Observe that $60P=(649,2654)$. Therefore $\lambda\equiv\frac{2653}{649} \mod 5959$. Since $649 \not\equiv 0 \mod 5959$, we encounter a contradiction to the group law because any element $P_i \in S$ must have a unique inverse and a slope $\lambda\equiv\frac{2653}{649} \mod 5959$ will produce $\mathcal{O}$. Our conclusion is twofold: first, $(S,+)$ is not a group over $\mathbb{Z}/5959\mathbb{Z}$, and second, $\gcd(649,5959)\neq1$. Since $\gcd(649,5959)=59$ and $1<59<5959$, $59$ is a nontrivial factor of $5959$.

\indent The elliptic-curve Diffie-Hellman key exchange protocol can be implemented on the group $(E(\mathbb{Z}/p\mathbb{Z}), +)$ as follows:
\begin{enumerate}
\item Alice and Bob publicly agree on a prime $p$ and an elliptic curve $S$ over $\mathbb{Z}/p\mathbb{Z}$. They then agree on a point $P \in S(\mathbb{Z}/p\mathbb{Z})$.
\item Alice chooses a private key $m$ and sends Bob $mP$.
\item Bob chooses a private key $n$ and sends Alice $nP$.
\item Alice and Bob both compute $mnP$, their shared secret key.
\end{enumerate}
It should be noted that elliptic-curve Diffie-Hellman is not a full cryptosystem, only a protocol to agree on a shared secret key that can be used for encryption by some other method.

\indent For example, Alice and Bob publicly agree on $p=571$ and the elliptic curve $S$ given by $y^2=x^3-12x+5$. They then agree on $P=(16,324)$. Alice privately computes and sends Bob $39P=(148,387)$. Bob privately computes and sends $121P=(465,556)$. Alice privately computes $39(465,556)=(202,445)$, and Bob privately computes $121(148,387)=(202,445)$. The shared secret key is $39(121P)=121(39P)=(202,445)$.

\indent The security of an elliptic curve cryptosystem is dependent on the solution to the elliptic curve discrete logarithm problem. In other words, if $S$ is an elliptic curve over $\mathbb{Z}/p\mathbb{Z}$ and $P \in S(\mathbb{Z}/p\mathbb{Z})$, and given $Q$ a multiple of $P$, we aim to find $n\in\mathbb{Z}$ such that $nP=Q$. The naive approach is to simply check each possible value of $n$ until one arrives at the solution $Q$. However, the naive approach becomes computationally infeasible as $p$ is sufficiently large. Currently, it appears that the discrete logarithm problem on $E(\mathbb{Z}/p\mathbb{Z})$ is more difficult than the discrete logarithm problem on $\mathbb{Z}/p\mathbb{Z}^*$. Therefore, an elliptic curve cryptosystem can offer equivalent security to many cryptosystems currently in use with much smaller numbers, which allows for great gains in efficiency. The smaller numbers involved in an elliptic curve cryptosystem require less memory and are less computationally intensive as compared to many current alternatives.

\indent Elliptic curve cryptography has become much more prevalent in recent years. Presently, Bitcoin uses an elliptic curve digital signature algorithm to verify coin ownership. Additionally, in 2015 the Line Corporation implemented elliptic-curve Diffie-Hellman in its messaging application, Line, in order to apply end-to-end-encryption to all messages. Google also utilizes elliptic curve public keys for encryption and verification. Elliptic curves are crucial cryptographic tools given their mathematical structure and properties.

\end{document}